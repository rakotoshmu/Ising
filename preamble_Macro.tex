% notations textuelles condensées
	\newcommand{\ssi}{si, et seulement si, }
	\newcommand{\ps}{\ensuremath{\text{ p.s.}}}



% restriction d'applications
	\newcommand{\restrictionaux}[2]{{#1\,\smash{\vrule height .8\ht1 depth .85\dp1}}_{\,#2}}
	\newcommand{\restr}[2]{\ensuremath{\mathchoice
		{\setbox1\hbox{${\displaystyle #1}_{\scriptstyle #2}$}
		\restrictionaux{#1}{#2}}
		{\setbox1\hbox{${\textstyle #1}_{\scriptstyle #2}$}
		\restrictionaux{#1}{#2}}
		{\setbox1\hbox{${\scriptstyle #1}_{\scriptscriptstyle #2}$}
		\restrictionaux{#1}{#2}}
		{\setbox1\hbox{${\scriptscriptstyle #1}_{\scriptscriptstyle #2}$}
		\restrictionaux{#1}{#2}}}}



% notations mathématiques
	% applications
		\newcommand{\map}[4]{\ensuremath{\begin{array}{r c l}#1&\to&#2\\#3&\mapsto&#4\end{array}}}
		\newcommand{\mapp}[6]{\ensuremath{\begin{array}{r c l}#1&\to&#2\\#3&\mapsto&#4\\#5&\mapsto&#6\end{array}}}

	% listes d'éléments
		\newcommand{\set}[1]{\ensuremath{\left\lbrace #1 \right\rbrace}}
		\newcommand{\scal}[2]{\ensuremath{\left<#1,#2\right>}}
		\newcommand{\couple}[2]{\ensuremath{\left(#1,#2\right)}}
		\newcommand{\seq}[2]{\ensuremath{\displaystyle{\left(#1\right)_{#2}}}}

	% notations d'Euler
		\newcommand{\integrale}[2]{\ensuremath{\displaystyle{\int_{#1}^{#2}}}}
		\newcommand{\somme}[2]{\ensuremath{\displaystyle{\sum_{#1}^{#2}}}}
		\newcommand{\produit}[2]{\ensuremath{\displaystyle{\prod_{#1}^{#2}}}}
		\newcommand{\union}[2]{\ensuremath{\displaystyle{\bigcup_{#1}^{#2}}}}
		\newcommand{\inter}[2]{\ensuremath{\displaystyle{\bigcap_{#1}^{#2}}}}
		\newcommand{\sommedirecte}[2]{\ensuremath{\displaystyle{\bigoplus_{#1}^{#2}}}}
		\newcommand{\tensoriel}[2]{\ensuremath{\displaystyle{\bigotimes_{#1}^{#2}}}}
		\newcommand{\cartesien}[2]{\ensuremath{\displaystyle{\bigtimes_{#1}^{#2}}}}

	% normes
		\newcommand{\abs}[1]{\ensuremath{\left| #1 \right|}}
		\newcommand{\norm}[1]{\ensuremath{\left\| #1 \right\|}}

	% probabilités
		\newcommand{\univers}{\ensuremath{\left(\Omega, \mathcal F, \mathbb P \right)}}
		\newcommand{\proba}[1]{\ensuremath{\mathbb P\left(#1\right)}}



% mise en forme
	\newcommand{\notion}{}



% préférences personnelles de typographie
	\renewcommand{\epsilon}{\varepsilon}
	% \renewcommand{\phi}{\varphi}
	\renewcommand{\tilde}{\widetilde}
	\renewcommand{\hat}{\widehat}
	\renewcommand{\bar}{\widebar}



% opérateurs mathématiques
	\DeclareMathOperator{\id}{id}
	\DeclareMathOperator{\Ran}{Im}
	\DeclareMathOperator{\Ker}{Ker}
	\DeclareMathOperator{\Tr}{Tr}
	\DeclareMathOperator{\supp}{supp}
	\DeclareMathOperator{\Vect}{Vect}
	\DeclareMathOperator{\pr}{pr}
	\DeclareMathOperator{\Int}{Int}
	%\DeclareMathOperator{\ln}{ln}



% théorèmes
	\theoremstyle{plain} %résultats
		\newtheorem{thm}{Théorème}[section]
		\newtheorem{cor}[thm]{Corollaire}
		\newtheorem{pte}[thm]{Propriété}
		\newtheorem{prop}[thm]{Proposition}
		\newtheorem{res}[thm]{Résultat}
		\newtheorem{lem}[thm]{Lemme}
	\theoremstyle{definition} %définitions
		\newtheorem{definition}[thm]{Définition}
	\theoremstyle{remark} %remarques
		\newtheorem*{rk}{Remarque}