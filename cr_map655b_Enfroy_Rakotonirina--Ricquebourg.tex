\documentclass[a4paper,11pt]{article}

\usepackage[utf8]{inputenc}
\usepackage[T1]{fontenc}
\usepackage[french]{babel}
% \usepackage{fullpage}
\usepackage{amsmath, amsfonts, amssymb, amsthm}
% \usepackage{mathabx}
% \usepackage{bbm}
\usepackage{stmaryrd}
% \usepackage{enumerate}

% notations textuelles condensées
	\newcommand{\ssi}{si, et seulement si, }
	\newcommand{\ps}{\ensuremath{\text{ p.s.}}}



% restriction d'applications
	\newcommand{\restrictionaux}[2]{{#1\,\smash{\vrule height .8\ht1 depth .85\dp1}}_{\,#2}}
	\newcommand{\restr}[2]{\ensuremath{\mathchoice
		{\setbox1\hbox{${\displaystyle #1}_{\scriptstyle #2}$}
		\restrictionaux{#1}{#2}}
		{\setbox1\hbox{${\textstyle #1}_{\scriptstyle #2}$}
		\restrictionaux{#1}{#2}}
		{\setbox1\hbox{${\scriptstyle #1}_{\scriptscriptstyle #2}$}
		\restrictionaux{#1}{#2}}
		{\setbox1\hbox{${\scriptscriptstyle #1}_{\scriptscriptstyle #2}$}
		\restrictionaux{#1}{#2}}}}



% notations mathématiques
	% applications
		\newcommand{\map}[4]{\ensuremath{\begin{array}{r c l}#1&\to&#2\\#3&\mapsto&#4\end{array}}}
		\newcommand{\mapp}[6]{\ensuremath{\begin{array}{r c l}#1&\to&#2\\#3&\mapsto&#4\\#5&\mapsto&#6\end{array}}}

	% listes d'éléments
		\newcommand{\set}[1]{\ensuremath{\left\lbrace #1 \right\rbrace}}
		\newcommand{\scal}[2]{\ensuremath{\left<#1,#2\right>}}
		\newcommand{\couple}[2]{\ensuremath{\left(#1,#2\right)}}
		\newcommand{\seq}[2]{\ensuremath{\displaystyle{\left(#1\right)_{#2}}}}

	% notations d'Euler
		\newcommand{\integrale}[2]{\ensuremath{\displaystyle{\int_{#1}^{#2}}}}
		\newcommand{\somme}[2]{\ensuremath{\displaystyle{\sum_{#1}^{#2}}}}
		\newcommand{\produit}[2]{\ensuremath{\displaystyle{\prod_{#1}^{#2}}}}
		\newcommand{\union}[2]{\ensuremath{\displaystyle{\bigcup_{#1}^{#2}}}}
		\newcommand{\inter}[2]{\ensuremath{\displaystyle{\bigcap_{#1}^{#2}}}}
		\newcommand{\sommedirecte}[2]{\ensuremath{\displaystyle{\bigoplus_{#1}^{#2}}}}
		\newcommand{\tensoriel}[2]{\ensuremath{\displaystyle{\bigotimes_{#1}^{#2}}}}
		\newcommand{\cartesien}[2]{\ensuremath{\displaystyle{\bigtimes_{#1}^{#2}}}}

	% normes
		\newcommand{\abs}[1]{\ensuremath{\left| #1 \right|}}
		\newcommand{\norm}[1]{\ensuremath{\left\| #1 \right\|}}

	% probabilités
		\newcommand{\univers}{\ensuremath{\left(\Omega, \mathcal F, \mathbb P \right)}}
		\newcommand{\proba}[1]{\ensuremath{\mathbb P\left(#1\right)}}



% mise en forme
	\newcommand{\notion}{}



% préférences personnelles de typographie
	\renewcommand{\epsilon}{\varepsilon}
	% \renewcommand{\phi}{\varphi}
	\renewcommand{\tilde}{\widetilde}
	\renewcommand{\hat}{\widehat}
	\renewcommand{\bar}{\widebar}



% opérateurs mathématiques
	\DeclareMathOperator{\id}{id}
	\DeclareMathOperator{\Ran}{Im}
	\DeclareMathOperator{\Ker}{Ker}
	\DeclareMathOperator{\Tr}{Tr}
	\DeclareMathOperator{\supp}{supp}
	\DeclareMathOperator{\Vect}{Vect}
	\DeclareMathOperator{\pr}{pr}
	\DeclareMathOperator{\Int}{Int}
	%\DeclareMathOperator{\ln}{ln}



% théorèmes
	\theoremstyle{plain} %résultats
		\newtheorem{thm}{Théorème}[section]
		\newtheorem{cor}[thm]{Corollaire}
		\newtheorem{pte}[thm]{Propriété}
		\newtheorem{prop}[thm]{Proposition}
		\newtheorem{res}[thm]{Résultat}
		\newtheorem{lem}[thm]{Lemme}
	\theoremstyle{definition} %définitions
		\newtheorem{definition}[thm]{Définition}
	\theoremstyle{remark} %remarques
		\newtheorem*{rk}{Remarque}

\title{Projet de simulations aléatoires : Modèles d'Ising}
\author{Aurélien Enfroy, Shmuel Rakotonirina{-}-Ricquebourg}

\begin{document}
\maketitle

\section{Implémentation du modèle d'Ising}

On rappelle la définition du modèle d'Ising sur un réseau carré :
\begin{definition}
On fixe $C$ le réseau carré de dimension 2 de taille $N^2$. Le modèle d'Ising est la distribution sur l'espace d'état $\set{\pm 1}^C$ dont la loi est donnée par
$$\forall x \in \set{\pm 1}^C, \pi(x) = \frac{1}{Z_T} \exp \left( \frac{\somme{u\sim v}{} J_{u,v} x_{u} x_{v} + \somme{u}{} h_{u} x_{u}}{T} \right)$$
où $T>0$ est appelée la température, $Z_T$ est une constante de normalisation, $J_{u,v}$ est la force d'interaction entre $u$ et $v$ et $h_{u}$ est le champ magnétique extérieur en $u$.
\end{definition}

Pour l'implémentation, on remarque qu'il n'y a pas besoin du paramètre $T$, qu'on peut compter dans $J$ et $h$. On représente alors ces paramètres en prenant $x \in \mathcal M_{N,N}(\set{\pm 1})$, $h \in \mathcal M_{N,N}(\mathbb R)$ et $J$ comme une matrice à trois entrées $\tilde J \in \mathcal M_{N,N,2}(\mathbb R)$ où
$$\tilde J_{i,j,1} = J_{(i,j),(i+1,j)} \text{ et } \tilde J_{i,j,2} = J_{(i,j),(i,j+1)}.$$

\section{Simulation naïve en petite taille}

Pour vérifier que les algorithmes fonctionnent, on les compare à la méthode naïve en petite taille. Pour cela, on numérote les $2^{N^2}$ états (dans l'ordre lexicographique en lisant les matrices colonne par colonne).

\section{Simulation par l'échantilloneur de Gibbs}

En reprenant les notations du cours, on a pour $u \in C$ et $x \in \set{\pm 1}^C$
\begin{align*}
\pi_u(x_u \mid x^u)
&= \frac{\pi(x_u,x^u)}{\pi(1,x^u) + \pi(-1,x^u)}\\
&= \frac{e^{x_u \lambda_u}}{e^{\lambda_u} + e^{-\lambda_u}}
\end{align*}
où $\lambda_u \doteq \somme{v \sim u}{} J_{u,v} x_v + h_u$. Donc $\pi_u(\cdot \mid x^u) = \mathcal B(\frac{e^{\lambda_u}}{e^{\lambda_u} + e^{-\lambda_u}}) = B(\frac{1}{1 + e^{-2\lambda_u}})$

\end{document}